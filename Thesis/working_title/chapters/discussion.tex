\section{Answering Research Questions}
% Discussion
\subsection{Research Question 1} 
Is it possible to get real-time access to the data we record with the Empatica E4?\\[10pt]
Based upon the findings of the pilot experiment we conducted in section \ref{pilot}, we are able to answer in the affirmative. Using the MATLAB \gls{tcp} client we implemented in the scope of this project we were able to derive raw signal data from the Empatica E4, using a wireless Bluetooth connection, at the same rate it was recorded.

\subsection{Research Question 2}  
Is it possible to detect and distinguish different degrees of mental workload, as well as two emotional states, such as pleasant and unpleasant, using common machine learning techniques on three physiological signals that were recorded with the Empatica E4 in an experimental setting?\\[10pt]
%\ref{aadb}
With the methods we described in \ref{datainterpretation} we were able to classify a total of 6 different emotional states, with accuracies as high as 58\% for individual classifiers, and an average of 44 \% for all machine learning algorithms. These values, were derived from individual and collective algorithm performances using the original feature set, with a total of 21 features that were extracted from \gls{hrv}, \gls{gsr} and skin temperature.\\
Further, we were able to achieve even higher performances for binary classification tasks, aiming to identify specific emotional states. Considering the average performances of all four algorithms we achieved accuracies of 89\% for stress sessions, 72\% for visual stimulation sessions, and 91\% for resting intervals using again all 21 features.\\
Addressing the distinction of similar emotional states, we assessed the performances of all four classifiers using data sets, only comprised of session from either visual stimulation or stress sessions. Considering again the full feature set we achieved performances of 77\% for resting and stress conditions, and 45\% for both medium and high stress levels, as well as emotional states.\\
Lastly, we report feature dependent accuracies of 61-69\% for all of the abovementioned classification tasks. These values were computed using the average performances of all four classifiers on the respective feature sets.\\
Considering these results, we conclude that our methods are more sensitive to the presence or absence of a certain state opposed to small or gradient changes. But much like Byrne et al. (1996) we believe this type of "coarse" state assessment not necessarily to be a problem. The application of such a binary assessment of workload or emotion may still provide significant information to adaptive logic (e.g. on-task versus off-task) \cite{Byrne1996}. Therefore, we again answer in the affirmative and believe in the potential of our system for the neuroergonomic assessment in adaptive workplaces. 

\subsection{Research Question 3}  
Which algorithm is best suited for the classification task introduced in RQ2?

We found that overall \gls{svm} with 5-fold Crossvalidation performed best when used in combination the reduced time feature set. However, when compared over all feature sets, all four algorithms achieved similar results ranging from 65 to 67 \% on average.

\section{Success and Limitations of Neuroergonomic Assessment}
Although, we achieved accuracies well above chance for the classification of 6 different emotions using our selected features, and reached even higher performances on the binary classification of subsets using a variety of feature combination, we acknowledge that there is still room for improvement. Therefore, in the following section, we will discuss the limitations of the present and the possibilities of future work.

\subsection{Limitations}\label{lim}
\textbf{Signal Quality.}
In the sense of the concept of "Garbage In Garbage Out" the recent work focused heavily on the quality of data extraction and processing methods. The overall morphology and quality of the signals we derived, using the E4 wristband, were adequate and similar to other commercial devices. Further, we were not able to identify any drop-off in quality caused by the derivation from the standard measurement locations of \gls{ppg} and \gls{gsr}. We did however notice disproportional artifact levels in the \gls{bvp} signal during break intervals of our experiment, in which the restrictions to the participants movement were lifted. This suggests the susceptibility of wrist based signal detection to motion artifacts caused by muscle activity and subsequent sensor displacement. Even though these implications cast an unfavorable light on the E4, we could argue that this is an inherent problem to all optical sensors. However, at this point and without further investigation we are not able to determine the effects on future applications.\\[10pt]
%\section{suitability of E4 - Motion Artifacts}
%overall a good device but the reliance on optical sensors still causes a lot of motion artifacts. expecially the large disturbances caused even by miniscule muscle tremors of fingers could lead to long intervals of measurement that cant be used for classification. This requires an elaborate data collection system and extremely strict pre-processing, as well as signal admission.
%We propose a more adaptive algorithm that recognizes motion artifacts and instead of filtering them shifts the weight of the signals in the classification algorithm in a way that if a signal suffers from heavy artifacts in a certain time interval its influence is reduced by adjusting the weight and therefor other signal tribute more to the classification instead..until no artifact is recognized for a certain period of time. If we consider the model of fusion from the chinese guys that is
%Otherwise it would be wise to explore contact free forms of emotion recognition, although their implementation may present some obstacles too.
%\textbf{BVP Peak Detection Algorithm.}
%we observe some issues in one subject with our pre processing of bvp, due to a extreme reflection peak on every pulse wave causing false peak detection and dismissal of the data. This gives us reason to believe that there is still potential to improve. We can not assess the probability of such a failure because of the limited subject group. Or if in some way it was only circumstantial.
%However in the filter function for suitable data segments for feature extraction, based on condition A and B proofed to work well and prevented for faulty data to enter the ML algorithm therefore upholding its integrity.\\
%\textbf{Feature Selection.}
%As expected we were able to observe the influence of feature selection on classification accuracy among all learning algorithms. 
%But we were able to achieve good result with a reduced set of features and for some of the binary classifications even with only gsr and temp features.
%We propose to try the method of combining a lot of simple binary classifiers and fuse them together in a linear fashion (like the Chinese fellows did) quote paper and reference their results
%Create additional features is also possible, see (one of the three studies in related work, i think it was Picard) which achieved great results this way
%no geometrical features 
%The major advantage of geometric methods lies
%in their relative insensitivity to the analytic quality of
%the series of NN intervals[20]. The major disadvantage is
%the need for a reasonable number of NN intervals to
%construct the geometric pattern. In practice, recordings
%of at least 20 min (but preferably 24 h) should be used
%to ensure the correct performance of the geometric
%methods, i.e. the current geometric methods are inappropriate
%to assess short-term changes in HRV.\\
\textbf{Crossvalidation.} We have already touched upon the use of k-fold Crossvalidation as a way to increase the performance of machine learning algorithms.
However, our results do not clearly indicate the benefits of different degrees of Crossvalidation. Again, we propose that the limited number of data points to be a possible reason for this phenomenon. Thereby, using an excessive degree of k-fold Crossvalidation, which causes the individual validation sets to shrink and ultimately a distortion in the results, will lead to sub par or worse performances. Therefore, we would expect to observe further improvement by extending the data base and reassess k-fold parameters. One possibility could be to use the Leave One Out method. It is a rather extreme form of k-fold Crossvalidation, in which each sample is used once as a test set (singleton) while the remaining form the training set. We applied this method in the earlier stages of development, and very quickly ruled it out due to its extremely high computational cost. To conclude, even if the benefits of k-fold Crossvalidation were not immediately noticeable in our experiment its general usefulness and potential in future applications should be considered.\\[10pt]
\textbf{Evaluation Metric.}
In the scope of this thesis we used accuracy as a primary metric for algorithm evaluation. Although this is common practice, accuracy is known to be deceptive in some cases. For example, training learning algorithms on asymmetrical data sets. As most of our binary classification tasks featured a very asymmetrical data base (i.e. one of the emotional states was heavily favored), we faced some complications in the learning process. Especially during the classification of visual stimulation sessions in the complete data set. We achieved very similar performances for all algorithms across all feature sets, which lead to doubts in the validity of the results. Therefore, we employed additional metrics such as the Confusion Matrix, which indicates the number of correct and incorrect decisions for each of the target labels. Applying this method revealed faulty decision patterns for some of the algorithms on the above mentioned data set, which suggested some issues in the learning process. We believe that the small sample size in combination with the strong imbalance of the data set are the most likely causes for these results and that the best approach would be to conduct further experiments, creating a larger data set with more equally distributed states. \\[10pt] 
\textbf{Sample Size.}
Sample size, or the size of the data set, is known to be a great influence to the success of machine learning algorithms. Considering the algorithms that were selected for this project, we were able to witness the effects of sample size first hand. For example we noticed that algorithms such as Random Forest, Support Vector Machines, and Neural Networks, which are all known to be better suited for medium to large data sets, significantly underperformed on most of the smaller sub sets (i.e. SvS, EvE) when compared to the larger subsets (i.e. SvA, EvA, RvA). Also, the k-NN classifier, which is commonly known to perform well even on thinner data sets, performed exceptionally compared to the others on distinguishing among similar emotional states (i.e. SvS, EvE). Considering these results, we believe that the acquisition of further data could greatly improve the performance of some of the more complex learning algorithms and our system overall.\\[10pt]
\textbf{Emotion Elicitation.}
According to our initial approach we used visual stimulation to elicit two emotional states in our subjects: pleasant and unpleasant. As described in \ref{visstim} we presented a selected set of images, taken from the \gls{iaps}, for each emotion. Image selection was based on rating according to a two dimensional model (arousal and valence). In addition, subjective ratings (which used the same metric) on all presented images were gathered for all subjects directly after the visual stimulation sessions. After careful evaluation of the subjective ratings (supported by our observations during the experiment) we detected a number of discrepancies. Meaning that the emotion participants perceived from a certain picture would deviate from the one we intended to display. This misconception raised our concern due to its inevitably effects on the quality of the features we derived from the respective sessions. But, because this phenomenon was limited to a very small percentage of data points and the fact that our subject group was already very limited in size, we hesitated to filter out the flawed sessions. It is not clear to which extent the results of the respective classification tasks (i.e. EvA, EvE) were affected by this decision. On the other hand, it could be argued that these findings were coincidental and would have been rendered insignificant in a larger group of people. Nevertheless, we propose a more elaborate emotion elicitation paradigm, which firstly separates neutral and pleasant stimuli and secondly features stimuli that are more representative of the respective emotions, for future iterations.\\[10pt]
\textbf{Qualification Criteria.}
As described in section \ref{exppar} we included a total of 14 healthy subjects in our experiment. All of which reported to be healthy and willing to partake in the experiment. In retrospective we identified the following issues with the subject admission process.
\begin{itemize}
\item Recruited from the student body of HTW Saar and SNNU, most of our subjects came from a scientific background and were already familiar with the laboratory and experimental procedures. In addition, some of the participants had already worked with the \gls{iaps}. We can only assume the effects this familiarity had on the emotion elicitation capability of our visual stimulation tasks.
\item Although all participants gave their confirmation, some stated afterwards that they either were not feeling well, suffered from bad mood, or a mild headache but still wanted to partake. Again we can not completely estimate the influence of this on the measurement, or rule out that there is any influence at all
\end{itemize}
\textbf{Classification Interval.}
As we were dedicated to follow the recommendations of the Task Force of The European Society of Cardiology and The North American Society of Pacing and Electrophysiology (1996) we used \gls{hrv} features that were extracted from 5 minute recordings in our emotion classification process. However, it is also stated that the necessary interval length lies only between 1 min for HF and 2 min for LF, as the recording interval should be at least 10 times the length of the wavelength of the lowest frequency component. Considering real life settings, the reduction in classification time we could provide with this method, could greatly benefit the responsiveness and the overall accuracy of adaptive automation. Furthermore, we argue that due to the E4's susceptibility to motion artifacts the classification on 5 minute intervals is far too impractical for real world applications. However, the effects on \gls{hrv} frequency features should be evaluated first before deploying such a method.
\subsection{Future Work}
In this section we present a list of possible additions to the recent work, that were derived during the evaluation process and could be the addressed in subsequent projects.

\textbf{Large Scale Data Collection.}
Over the course of this thesis we encountered a number of issues with the machine learning process (see \ref{lim}). Many of which could be resolved by significantly extending the data base. Therefore, we would like to propose a subsequent large scale experiment, to collect proof our hypothesis and for the further improvement of the system at hand. Next to a larger subject group, we identify the refinement of the admission criteria as well as the emotion elicitation paradigm as key components for such an endeavor.\\[10pt]
\textbf{HMI Focused Application.}
Unfortunately, we were unable to realize an experimental setting within an actual collaborative work environment in the given time frame. Therefore, we would like to see a continuation of our work which is focused on the development of adaptive logic based on the application of the current system in an collaborative work environment.\\[10pt]
\textbf{Motion Compensation.}
Our ultimate goal is the application of the recent system in a real life environment. Because of this we need to address problems that are likely to arise once we step outside of laboratory conditions. We consider motion artifacts to be one of the major obstacles to system implementation. Therefore we propose the investigation of motion based signal processing strategies, which attempt to adapt signal detection and analysis based on acceleration data derived from the E4's accelerations sensors.\\[10pt]
\textbf{Weighted Fusion Strategy.}
To our surprise single signal feature sets such as \gls{gsr} performed well on many of the classification tasks. Inspired by the work of Wei et al. (2018) we propose to investigate the effects of Weighted Fusion Strategy on the recent work.\\[10pt]
\newpage
\textbf{Linear Discriminant Analysis.}
Considering  the work of Maaoui and Pruski (2010) who achieved a classification accuracy of 92\% over 6 emotions by using \gls{svm} and Fisher Linear Discriminant analysis, in respect to the results of the present thesis, we were pondering the implications of adapting this strategy. However, due to the limited time frame we were unable to do so. Therefore, we would like to conduct further research on the matter in the future.

