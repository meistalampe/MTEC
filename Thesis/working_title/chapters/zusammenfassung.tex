
% Zusammenfassung

In den letzten Jahren sind Wearables zu einer festen Größe in unserer Gesellschaft geworden. In der Form von Smartwatches und Fitness-Armbändern haben Wearables Einzug in annähernd dreißig Prozent aller deutschen Haushalte gehalten. Neben den Grundfunktionen eines Telefons oder einer Uhr bieten diese praktischen Geräte eine Bandbreite an Funktionen. Viele dieser Zusatzfunktionen befassen sich mit dem Messen der Vitalparameter des Trägers zur Früherkennung von Krankheiten oder zur Optimierung der sportlichen Leistungsfähigkeit.
Forschungsarbeiten der letzten Jahre lassen das Potential der Anwendung von psychophysiologischem Monitoring in besonders stressvollen Arbeitsplätzen erahnen, ein Gebiet in dem Wearables nur selten anzutreffen sind.
Aus diesem Grund haben wir ein System für Arbeiter an besonders stressvollen Arbeitsplätzen, insbesondere collaborative Arbeitsplätze mit Mensch-Roboter-Interaktion (HRI). Das System basiert auf einem Wearable, welches am Handgelenk getragen wird, und ist in der Lage den mentalen Zustand des Trägers zu beurteilen.
Durch den Einsatz eines solchen Systems sind wir in der Lage neuroergonomische Atbeitsplätze zu schaffen die sich an die cognitive Leistungsfähigkeit eines Arbeiters anpassen und somit den Einfluss von Stress, als einen der führenden Gründe für Krankheiten und Verletzungen unter der arbeitenden Bevölkerung, drastisch zu senken.