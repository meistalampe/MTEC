
% Abstract
In recent years wearables have become a staple in our society. In the form of Smartwatches and fitness-tracking devices wearables have made their way into nearly a third of german households. Aside from the basic functions of a phone or a watch these devices offer a wide array of functionalities, many of which revolve around monitoring the user's vital parameters for health reasons as well as improving their physical performance in recreational activities. \\
Research suggests that there is great potential in the application of psycho-physiologic monitoring in highly demanding workplaces of today's industry, an area in which wearables are still largely underrepresented. \\
Therefore, we built a new system based on a wrist worn device capable of monitoring workers in stressful workplaces, specifically collaborative workplaces involving Human-Robot-Interaction (HRI), and predicting their current mental state on a real-time basis. 
With the deployment of such a system we are able to create neuroergonomic workplaces that are sensitive to a persons mental capability and capable of eliminating stress as one of the leading causes of injury and disease in the working population.

