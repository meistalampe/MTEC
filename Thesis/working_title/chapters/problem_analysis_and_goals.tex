
% Problem Analysis and Goals
\section{Psychophysiological Framework}
To fully understand the role of psychophysiologic measurement in adaptive automation we will take a look at the theoretical frameworks behind it. However, providing a complete overview on this topic would be far too extensive for the scope of this thesis. Therefore, we will only give a short summary of the work done by Byrne and Parasuraman (1996).\\
The application of physiological measures in adaptive automation is built on the premise that there is indeed an ideal mental state for human operators in a given task environment and that any deviation from this state would be detectable in the measurement. 
This hypothesis is based on resource and capacity theories of information processing, which suggest that humans draw from a limited pool of resources whenever they process information \cite{Byrne1996}. Over the years, many researcher delivered evidence for a connection between this resource utilization and physiological measures of activation, therefore establishing the importance of psychophysiological measures in the field of adaptive automation.\\
%For instance, while both cognitive and compensatory effort were associated with physiological measures, psychological and physiological strain, arising from mental overload or underload, were detected in psychophysiological measurements. \\
However, psychophysiological measures perform a dual role in adaptive automation systems. First, there is the investigatory role, which is often referred to as the developmental approach. This approach is focused on using the information psychophysiological measures provide on the mechanisms underlying performance changes corresponding to changes in automation, and further the development of model-based and hybrid approaches \cite{Byrne1996}. The second role, is often characterized as the regulatory approach. Here, unique information about the human operator is gathered from psychophysiologic measurements. This information is then used as input to a hybrid adaptive logic, thus allowing for dynamic restructuring of the task environment. Although, this approach seems ideal to support the operation of an adaptive system due to its the immediate effect on the automated work environment, there may be years of effort and considerable maturation in technology required for it to be efficient in its application.  

\section{Psychophysiological Measures}
The identification of suitable psychophysiological measures plays a vital role to the success of an adaptive work environment. Considering the dual role framework of psychophysiology, there is a distinction to be made between the two applications in adaptive automation. Because the developmental approach is in alignment with the majority of applications in psychophysiological research, the often stated criteria of specificity, diagnosticity, and intrusiveness for selecting workload assessment techniques also hold for adaptive automation \cite{Byrne1996}. On the other hand, criteria for the regulatory role of psychophysiology in adaptive automation have to be more strict. As they become part of closed-loop systems operating in real-time their potential impact is far greater, and their effects more immediate compared to when used for developmental measures.  
In addition, the cost in terms of intrusiveness and technical requirements have to be weighed against the explanatory power of a certain measure. If the gain in predictive value does not offset the cost of implementation, a measure is not considered for applications outside of laboratory environment.\\
As the recent work is determined to employ the Empatica E4 wristband, we are limited to the measures that are provided by this platform. These measures are blood volume pulse (\gls{bvp}), skin response (\gls{gsr}), and surface temperature.

\section{Photoplethysmography}
Photoplethysmography \gls{ppg} is an optical measurement technique, used to detect blood volume changes in the microvascular bed of tissue \cite{Allan2007}. To work PPG only requires a few opto-electronic components. First, a light source is used to illuminate the tissue. Then a photodetector measures the variations in light intensity associated with changes in perfusion in the catchment area. 
The most common light sources in PPG produce wavelengths in the red or near infrared area. This specific part of the spectrum, also referred to as the optical water window, is chosen for its ability to pass through biological tissue with relative ease. Therefore, influences associated with light-tissue interactions are widely reduced and the measurement of blood flow or volume is facilitated at these wavelengths.

\subsection{The PPG waveform: characteristics and analysis} 

The PPG waveform is comprised of two major components. The pulsatile component, often referred to as the "AC" component, possesses a fundamental frequency of approximately 1 Hz, and it represents the increased light attenuation associated with the increase in microvascular blood volume with each heartbeat \cite{Allan2007}. It is superimposed onto the much larger "DC" component, which relates to the tissue and the average blood volume contained in the observation area. Variations in the DC component are slower and caused by respiration, vasomotor activity and vasoconstrictor waves, as well as thermoregulation \cite{Allan2007}.\\ 

[insert: ppg_raw.jpg, source: \cite{Allan2007}]

Its synchronization with the heart beat makes the AC pulse of the PPG waveform a valuable source of information on heart functions and condition. Based on the appearance of the AC pulse, two phases have been defined, reflecting its two most important properties. The first was labeled as the anacrotic phase and describes the rising edge of the pulse. This part of the waveform is primarily related to the systole. The second phase, that shows the effects of diastole and wave reflections from the periphery of the vascular system, can be observed in the successive falling edge of the pulse. This phase is called catacrotic.






%\section{Related Work}
% outline
% https://books.google.de/books?hl=de&lr=&id=9ERRDAAAQBAJ&oi=fnd&pg=PA239&dq=neuroergonomics+adaptive+automation&ots=bvcjBrmrib&sig=xR__KHbEoDxTgcRrmyzaz6mUuX8#v=onepage&q=neuroergonomics%20adaptive%20automation&f=false

