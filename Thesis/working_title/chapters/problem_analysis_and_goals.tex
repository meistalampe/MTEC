
% Problem Analysis and Goals
\section{Psychophysiology}
\subsection{Theory}
To fully understand the role of psychophysiologic measurement in adaptive automation we will take a look at the theoretical frameworks behind it. However, providing a complete overview on this topic would be far too extensive for the scope of this thesis. Therefore, we will only give a short summary of the work done by Byrne and Parasuraman (1996).\\
The application of physiological measures in adaptive automation is built on the premise that there is indeed an ideal mental state for human operators in a given task environment and that any deviation from this state would be detectable in the measurement. 
This hypothesis is based on resource and capacity theories of information processing, which suggest that humans draw from a limited pool of resources whenever they process information \cite{Byrne1996}. Over the years, many researcher delivered evidence for a connection between this resource utilization and physiological measures of activation, therefore establishing the importance of psychophysiological measures in the field of adaptive automation.\\
%For instance, while both cognitive and compensatory effort were associated with physiological measures, psychological and physiological strain, arising from mental overload or underload, were detected in psychophysiological measurements. \\
However, psychophysiological measures perform a dual role in adaptive automation systems. First, there is the investigatory role, which is often referred to as the developmental approach. This approach is focused on using the information psychophysiological measures provide on the mechanisms underlying performance changes corresponding to changes in automation, and further the development of model-based and hybrid approaches \cite{Byrne1996}. The second role, is often characterized as the regulatory approach. Here, unique information about the human operator is gathered from psychophysiologic measurements. This information is then used as input to a hybrid adaptive logic, thus allowing for dynamic restructuring of the task environment. Although, this approach seems ideal to support the operation of an adaptive system due to its the immediate effect on the automated work environment, there may be years of effort and considerable maturation in technology required for it to be efficient in its application.  

\subsection{Measures}







%\section{Related Work}
% outline
% https://books.google.de/books?hl=de&lr=&id=9ERRDAAAQBAJ&oi=fnd&pg=PA239&dq=neuroergonomics+adaptive+automation&ots=bvcjBrmrib&sig=xR__KHbEoDxTgcRrmyzaz6mUuX8#v=onepage&q=neuroergonomics%20adaptive%20automation&f=false

