
% Problem Analysis and Goals
\section{Problem Oriented}
The ability to detect human emotion...super important in HMI.
From before we know that psychophysiological measures are an approach worth taking because they can access emotion directly , methods such as face recognition based emotion detection can be deceived more easily.
And although there have been some approaches using psychophysiology measure in combination with ML that achieved significant scores..some problems remain.
They used clean data sets, stationary measures, one subject only and so on
We will attempt to evaluate measurements from multiple subjects with realistic data from the same device we could actually apply
\section{Aim, Objective, and Goals}
The objective of this thesis is mainly engineering a well functioning system.
Therefore we first need to establish a stable datastream of raw data. to be able to even  be able to access the physiological measures.
After this the next goal is to develop an algorithm for preprocessing that allows valid feature extraction.
Choosing the right features.
Applying ML methods. i.e. emotion recognition pipeline

\section{Research Questions}
RQ1: Confirmation of suitability of the Empatica E4 for real time measurement.

RQ2: Is it possible to detect different degrees of workload using standard machine learning techniques on extracted features from the Empatica E4?

RQ3: Is it possible to make a distinction between to emotional states?
%\section{Related Work}
% outline
% https://books.google.de/books?hl=de&lr=&id=9ERRDAAAQBAJ&oi=fnd&pg=PA239&dq=neuroergonomics+adaptive+automation&ots=bvcjBrmrib&sig=xR__KHbEoDxTgcRrmyzaz6mUuX8#v=onepage&q=neuroergonomics%20adaptive%20automation&f=false

