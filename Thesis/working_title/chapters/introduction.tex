
% Introduction
We live in the age of automation. In fact automation, in one form or another, has been a dominating force in almost every branch of industry for what has been the better part of a century now. It can be defined as the creation and application of technology by which the production and delivery of various goods and services is controlled and monitored with minimal human assistance. Automation can be encountered in many different forms, from a simple automatic control loop, that compares a measured value of a process with a desired set value to generate a regulating feedback, up to an artificial intelligence that is used in cars to detect pedestrians and perform an emergency break if necessary. 
The traditional form of automation is static automation. In this form, automation is an all-or-none technology either performing a task for us or not \cite{Byrne2006}. Introducing static automation into a work environment causes a significant change in roles. Machines exceed their former role as tools and the workers, once active operators, become passive monitors. Eventually causing a change in quality and quantity of the mental workload of the workforce. 
Although, there are numerous benefits, such as relieving workers from the strain of performing repetitive tasks, there is increasing evidence that static automation comes at the price of impaired decision making, manual skill degradation, loss of situational awareness, and monitoring inefficiency \cite{Byrne2006}. 
Recent studies by Molloy and Singh (1993) as well as Parasuraman, Mouloua and Molloy (1994), testing subjects in a multi-task flight simulator with optionally automatable components, reported a decrease of human operator detection of automation failures in a static automation scenario with constant task assignment of both operator and the system \cite{Byrne2006}.
In general a statically automated system still relies heavily on human operator monitoring to be flawless, especially if the automation quality is insufficient. 


This gave rise to new ideas and models of automation, breaking permanent task assignment and some other crap...^^


- mention study [check]  then transition to AA

% game plan
% - talk about risks of automation in relation to worker safety...they get bored  and get careless..boom
% - talk about adaptive automation, what is it? what forms/approaches exist? How would it help?
% - end with a statement that shows that adaptive automation is a better way of handeling things but there are still aspects that can be optimized
% - transition to neuroergonomics, explain what it is and what it does
% - show why we need it to optimize workflow and worker health
% - transition to methods and research that is done in the lab
% - mention that these approaches are not applicable to real life scenarios because of limitations in data acquisition etc
% - transition to psychophysiological measures..explain how the could be superior because the show real time reaction, that are less affected by masking behaviour and easier to obtain
% - explain that for a system to be able to be deployed it also has to be accepted by the workers (comfort)
% conclude: a mobile system , based on psychophysiological measures, comfortable
% explain that this is what we built and then briefly describe steps..what device? what measures?data processing algorithm..what is special about it? testing with different scenarios and machine learning stategies.

% Note: maybe fit in stress somehow with the neuroergonomics theme

%\section{Motivation}
%Physical illness, distress and injury are all well known to be possible consequences of a stressful workplace. 
%Therefore, stress management not only has become a key component in today's industry, but also a common research topic in recent years. Where disciplines such as organizational psychology attempt to address this matter by offering practical guidelines for the assessment and mitigation of workplace stressors, neuroergonomics merges the disciplines of neuroscience and ergonomics to provide for a deeper understanding of the neural bases of perceptual and cognitive functions in relation to technologies and settings in the real world. 
%In this thesis, we take a neuroergonomic approach to optimize stress management in collaborative workplaces by regulating the information flow of a human-robot-interface (\gls{hri}) according to changes in the mental state of the user. 
%Using a commercial wristworn device in combination with a processing unit, we created a system that is capable of obtaining and interpreting psycho-physiological information in real-time.
%We believe that the application of such a system could be a potent tool in risk management and therefore be greatly beneficial for the overall reduction of stress related incidents at collaborative workplaces.


\section{Acknowledgments}

