% Materials and Methods
This chapter elaborates on all working steps that were necessary to build, test, and evaluate the system and all of its components. First, the methodology we used to solve the research question is explained. In the Material section, all hardware and software that were used during the development process are listed. Afterwards, the Method section examines the data pipeline (i.e. data extraction, data pre-processing algorithm, and feature extraction), the experiment setup we used to build the database, and finally machine learning algorithms and evaluation methods.
\section{Research Methodology}
\textbf{RQ 1:} Is it possible to get real-time access to the data we record with the Empatica E4?\\[10pt]
We approached RQ1 by researching the E4's data streaming functionalities. We discovered that there was a possibility to transmit raw-data from the E4 to a Windows computer by using a streaming server application in combination with a specific USB Bluetooth receiver. The streaming server App is a developer tool provided by Empatica and can be used to register and pair, one or more Empatica devices to a certain PC. Once the pairing process is completed it is possible to access the data stream of a paired E4 device using a TCP client. We then conducted a web research on past projects involving data transmission to the Empatica streaming server. Among others we found the Empatica BLE Client for Matlab environment, developed at ICAT Virginia Tech. This TCP client was capable of storing the raw-data it received from the Empatica E4 in text files of the CSV-format. Based on these findings, we were confident to build our own TCP client, capable of providing real-time access to the raw signal data, using the Empatica web resources for developers (e.g. Documentation on message protocol, data streaming packets, and the E4 streaming server) in combination with the MATLAB environment. 
 
\textbf{RQ 2:} Is it possible to detect and distinguish different degrees of mental workload, as well as two emotional states, such as pleasant and unpleasant, using common machine learning techniques on three physiological signals (BVP, GSR, and skin temperature) that were recorded with the Empatica E4 in an experimental setting?\\[10pt]

Answering RQ2 required us to gather relevant data. We therefore conducted experiments in both, the Mindscan Lab at HTW Saar, as well as the Green Lab at the University Hospital Saarland. The recorded data was then used to evaluate the different machine learning algorithms we selected following the procedure we already described in \ref{mlsel}. 
%We therefore successfully conducted an experiment involving the measurement of BVP, GSR, and skin temperature of 14 subjects in [a] a relaxed state, [b, c] when completing two cognitive tasks, and [d, e] two tasks that were designed to elicit emotions using visual stimulation. We then used basic pre-processing methods, extracted statistical time features (mean, min, max, standard deviation) of all three signals, and a linear classifier for classification. Achieving results well above chance level we felt reassured and proceeded to develop a more elaborate data processing pipeline to provide a wide variety of high quality features to use our selected machine learning algorithms on.

\textbf{RQ 3:} Which algorithm is best suited for the classification task introduced in RQ2?\\[10pt]
The answer to RQ3 is based on the results of RQ2.
\section{Materials}
%All experiments and measurements were conducted within the facilities of Systems Neuroscience and Neurotechnology Unit, particularly the Green Lab, located at the University Hospital Saarland, and the Mindscan Lab, located at the HTW Saar (Technikum).

\subsection{Hardware}
\subsubsection{The Empatica E4}
The Empatica E4 wristband is a wearable wireless device designed for comfortable, continuous, real-time data acquisition. It is a class IIa medical device in the EU, according to CE Cert. No. 1876/MDD (93/42/EEC Directive) and was designed for daily life usage \cite{e4}.

\begin{figure}[ht]
	\centering
  \includegraphics[width=0.33\textwidth]{../images/E4overview.JPG}
	\caption{Overview of the Empatica E4 wristband.}
	\label{e4overview}
\end{figure}

Figure \ref{e4overview} shows an overview of the entire E4 wristband from either side indicating key attributes as wells as a total of four different sensors that will be discussed briefly in the following:

\begin{itemize}
\item \textbf{Photoplethysmography (\gls{ppg})} to provide blood volume pulse (\gls{bvp}), from which heart rate, heart rate variability and other cardiovascular features may be derived
\item \textbf{Electrodermal activity (\gls{gsr})} is used to measure sympathetic nervous system arousal and to derive features related to stress, engagement and excitement
\item \textbf{3-Axis accelerometer} to capture motion-based activity
\item \textbf{Infrared thermopile} for reading skin temperature
\end{itemize}

As the E4 is intended to be worn on the wrist these sensors are set up in a specific way to provide for optimal use. As can be seen on \ref{e4overview} the majority of the sensors are located on the backside of the main unit not including the \gls{gsr}-sensor, which is located on the wristband itself.\\
Wearing the E4 wristband is equally intrusive to wearing a watch and therefore providing a high level of convenience compared to other physiologic measures such as electrocardiogram \gls{ecg} or electroencephalogram \gls{eeg}.\\

\textbf{Sampling Specifications}\\[10pt]
All recordings were performed using only software licensed by Empatica. Using the approved streaming server application and the compatible Bluetooth dongle, the recorded data was streamed directly to an operator's personal computer via a Bluetooth connection. 

\textbf{EDA sensor}
\begin{itemize}
\item Sampling frequency: 4 Hz (Non customizable).
\item Resolution: 1 digit ~900 pSiemens.
\item Range: 0.01 $\mu$Siemens – 100 $\mu$Siemens.
\item Alternating current (8Hz frequency) with a
max peak to peak value of 100 $\mu$Amps (at 100
$\mu$Siemens).
\item Electrode(Placement): on the ventral (inner) wrist.
\item Electrode(Build): Snap-on, silver (Ag) plated with metallic core.
\item Electrode(Longevity): 4–6 months
\end{itemize}

\textbf{PPG sensor}
\begin{itemize}
\item Sampling frequency 64 Hz (Non customizable).
\item LEDs: Green (2 LEDs), Red (2 LEDs) Photodiodes: 2
units, total 15.5 $mm^{2}$ sensitive area.
\item Sensor output: Blood Volume Pulse (BVP) (variation
of volume of arterial blood under the skin resulting
from the heart cycle).
\item Sensor output resolution 0.9 nW / Digit.
\item Motion artifact removal algorithm: Combines different light wavelengths. Tolerates external lighting conditions.
\end{itemize}

\textbf{Infrared Thermopile}
\begin{itemize}
\item Sampling frequency: 4 Hz (Non customizable).
\item Range(Ambient temperature): -40...85$\deg$C (if available).
\item Range(Skin temperature): -40...115$\deg$C.
\item Resolution: 0.02$\deg$C.
\item Accuracy $\pm$0.2$\deg$C within 36-39$\deg$C.
\end{itemize}

\textbf{Real-time clock}
\begin{itemize}
\item Resolution(Recording mode): 5s synchronization resolution. Average of 6 seconds in 6 million seconds drift.
\item Resolution(Streaming mode): Temporal resolution up to 0.2 seconds with connected device.
\end{itemize}
\subsubsection{Processing Unit}
The processing unit it the expression given to the computer, which is used to run any software that is essential for data transmission and data processing, such as the Empatica streaming server application, MATLAB, and PyCharm. We used a Lenovo ThinkPad, with an Intel(R) Core(TM) i5-6200U, CPU @ 2.3 GHz, 8 GB RAM, running a 64-Bit version of Windows 10. 
\subsection{Software}
\subsubsection{E4 Streaming Server}
The E4 streaming server for Windows (version of May,2018) allows to forward real-time data of one or multiple Empatica E4 devices to one or multiple TCP socket connections. However, as each TCP connection is limited to receiving data from only one Empatica E4 device, the connection to multiple devices would also require multiple TCP connections. The E4 streaming server is intended to provide access to the data streams using scripts and applications \cite{E4SS}.\\

\begin{figure}[ht]
	\centering
  \includegraphics[width=0.75\textwidth]{../images/E4streamingServer.JPG}
	\caption[Illustration of the connectivity and function of the E4 streaming server]{Illustration of the connectivity and function of the E4 streaming server. On one side are E4s, connected over BTLE to the E4 streaming server using the BLED112 dongle. On the other side are TCP clients, connected to the E4 streaming server through TCP connections over the network. The lines originating from the E4s illustrate the data flow from the E4 through the E4 streaming server to the subscribed TCP client. For example, the data from the first E4 is forwarded by the E4 streaming server to the first and third TCP client. \cite{E4SS}}
	\label{e4ss}
\end{figure}

%Once the TCP clients are connected to the network, as shown in \ref{e4ss}, they are able to communicate with the E4 streaming server. This communication follows a specific message protocol that provides a general structure for client commands and server messages. In general, messages and commands are ASCII strings terminated with a newline character and encoded with UTF-8 \cite{E4MP}. 
\subsubsection{MATLAB}
MATLAB is a computing and visualization software package, published by MathWorks. It combines a desktop environment tuned for iterative analysis and design processes with a high level programming language for matrix-based mathematics. We used MATLAB (version R2015a) to create a TCP client that was deployed in our data extraction pipeline to send commands to, and receive messages from, the E4 streaming server. Thereby, allowing us to connect an E4 device and control data transmission via a simple user interface.
\subsubsection{PyCharm}
PyCharm is an integrated development environment (\gls{ide}) for the Python programming language. It is developed by the company JetBrains and provides easy access to a large collection of scientific tools, used for data analysis and visualization. We used PyCharm to manage all data related tasks, such as data extraction, data pre-processing, feature extraction, and machine learning implementation. 

\subsubsection{PsychoPy}
PsychoPy is an open-source package for running experiments in Python. PsychoPy combines the graphical strengths of OpenGL with the easy Python syntax to give scientists a free and simple stimulus presentation and control package. It is used for psychophysics, cognitive neuroscience and experimental psychology. We used PsychoPy to create and display the paradigm for our experiments.
\section{Methods}

\subsection{Participants}
% talk about composition of subject group
In total 14 subjects, of which 7 were male and 7 female, participated in the experiment. Subject age ranged from 24 to 50 years, with an average of 29 years. At the day of the experiment all participants reported to be feeling well and capable of partaking in the experiment. 
\subsection{Stroop Test}
\subsection{Visual Stimulation}
\subsection{Procedure}
One of the most important parts to this project was the collection of authentic data that could be used later on to develop a reliable classifier for our system. For that reason an experiment, specifically designed to elicit certain emotional and cognitive states in a subject, was conducted. The following section is focused on the procedure applied in this experiment.

The procedure was comprised of a total of five sessions. Every experiment was initiated with a short briefing session. Containing a short questionnaire, covering personal information of the participant as well as habits that may have a influence on the measurement. Further a series of questions, regarding their handedness, use and frequency of use of watches or other wearables was posed, to estimate the additional influence that may be caused by wearing the Empatica E4 wristband.
Concluding the first session, the participants were given a coarse outline of the experiment covering the structure and a basic description of their responsibilities.

The second session consisted of a baseline measurement used to log the participants form of the day and also to be able to account for environmental influences in the following processing steps. Before the start of the measurement the subject was placed on a chair in front of a monitor (24 inches, Resolution: 1080p) with a approximated distance of 1m. The Empatica E4 was then put on the wrist of the non-dominant hand and secured in a position that caused minimal light leakage to the PPG-sensor and provided optimal contact for the \gls{gsr} electrodes. After the participants were comfortable with the device a one minute test sequence was measured to verify the functionality of the system.
Consequently the paradigm was displayed on the monitor and the session was started. After reading the instructions, in which the subjects were asked to relax and remain still, and confirmation with the participant the measurement was initiated with a ten second countdown to give some additional time for preparation.
During the measurement the \gls{gsr}, \gls{bvp}, and temperature of the subject were measured for a duration of five minutes.
Afterwards, to conclude the second session, the participants had to give a subjective rating of their current mental state, regarding their stress level, ranging from 1 (completely relaxed) to 10 (stressed out).

The third session was comprised of three separate measurements, two cognitive tasks and a relaxation segment. As before a rating followed the recording. The ratings consisted of a subjective assessment by the subjects regarding their stress level. Additionally subjects had to rate the test difficulty on a scale from 1 (very easy) to 10 (very difficult) for both tasks. 
For the first measurement the subjects were instructed to count down aloud from 700 in steps of 7 while maintaining a certain pace. The counting rhythm was indicated by a flashing dot on the instruction screen, for a duration of five minutes. The dot's color and flashing frequency were altered during the experiment to further increase difficulty at the three and four minute mark. If the participants were to slow down or loose track an instructor would intervene to help.
The second task consisted of a Stroop-Word-Color test. The test featured 11 different colors, resulting in a total of 220 trials the subjects had to work through. Although the color palette seems rather extensive when compared to the standard 3 color variation of the test, this was a conscious decision to guarantee a test time of at least 5 minutes to mitigate monotony. Each trial presented the subject with a colored word in the center of the screen and one possible answer to either side. The participants then had to choose the right answer based on the color of the word.Each decision was recorded via a key press on the keyboard. 

%Wearing the device is as easy as wearing a watch.
%Wear the E4 with the case on top of your wrist. Wear it snugly, so that it does not move around,

%but not so tight that it is uncomfortable.
%Which side should you wear it on? Traditional recommendations are to record EDA on the nondominant
%side to minimize motion artifacts (e.g. a right-hander would wear it on their left wrist).
%However, recent studies show that the dominant side may have a much stronger EDA signal during
%certain kinds of stress. Also, neurological events (such as seizures) may elicit EDA on only one side.
%(For more information see: Picard, R. W., Fedor, S., & Ayzenberg Y., “Multiple Arousal Theory and Daily-
%Life Electrodermal Activity Asymmetry” Emotion Review, March 2015.). Depending on your purposes,
%you may want to measure on the right, left, or both wrists.
%The E4 button may be positioned on the same side as the thumb or on the other side – either
%orientation works fine.
%The EDA electrodes (under the snaps) should be on the inside of the wrist. You may optionally line
%them up with a finger, e.g. the third (ring) finger, but this is not required
%\begin{figure}[ht]
%	\centering
%  \includegraphics[width=0.33\textwidth]{../images/E4wearing.JPG}
%	\caption{A figure depicting the intended way the E4 wristband should be worn.}
%	\label{e4wearing}
%\end{figure}

\subsection{Signal Analysis}




\subsubsection{Heart Rate Variability}
% BVP PRE PROCESSING
% 	- PEAK DETECTION
%		- bandpass zero phase filtering
%		- clipping
%		- squaring
%		- moving averages
%		- thresholding
%		- blocks of interest
%		- peak detection
%		- peak validation
% INTER BEAT INTERVALS
%	- DETECT OUTLIER
%	- DETECT ECTOPIC BEATS
%	- ARTIFACT DECLARATION
%	- LINEAR INTERPOLATION
% DATA VALIDATION
% FEATURE EXTRACTION
% CLASSIFICATION AND EVALUATION

\subsubsection{GSR}
% GSR PRE PROCESSING
%	- low pass zero phase filtering
%	- smoothing: moving average filtering
% FEATURE EXTRACTION

\subsubsection{Temperature}
% TEMPERATURE PRE PROCESSING
%	- smoothing: moving average filtering
% FEATURE EXTRACTION