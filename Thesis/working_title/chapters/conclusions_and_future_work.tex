
% Conclusions
Our work was guided by our aspirations to facilitate neuroergonomic assessment in collaborative workplaces. To reach this goal, we took on the task of developing a truly unobtrusive monitoring system by employing the Empatica E4 to provide wireless acquisition of physiological data. We established a data base for machine learning by conducting a series of experiments, which were specifically designed to elicit authentic emotions, using the newly developed system in a lab environment. We implemented state-of-the-art signal processing and analysis to derive meaningful features from three different physiological channels (BVP, GSR, and skin temperature) and eventually utilized them in our investigation on suitable learning algorithms for emotion classification.\\
All things considered, we were pleased by the outcome and the system's overall performance under the given circumstances. We were able to deliver proof of concept for the system we developed and facilitate the neuroergonomic assessment in binary classification tasks. However, we acknowledge the fact that there is still room for improvement and further refinement is still in order to guarantee the success of our system in real life settings. We consider both, the extension of the data base, as well as the investigation of more elaborate emotion elicitation methods, the top priorities for subsequent projects.




